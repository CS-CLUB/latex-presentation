\documentclass[xcolor={dvipsnames}]{beamer}
\usetheme{CambridgeUS}
\usecolortheme{default}
\setbeamertemplate{blocks}[rounded][shadow=true] 
\setbeamertemplate{navigation symbols}{} 
%\addtobeamertemplate{headline}{%
%    \vskip1pt%
%    %\setbeamertemplate{footline}[page number]%
%    \setbeamercolor{page number in head/foot}{use=headline,fg=headline.fg,bg=headline.bg}
%    \usebeamertemplate{footline}%    
%}
\setbeamertemplate{footline}[page number]{}

%language
\usepackage[english]{babel}
%encoding
%linux
\usepackage[utf8]{inputenc}
%windows
%\usepackage[latin1]{inputenc}

%modern font layout
\usepackage[T1]{fontenc}
%allowing for multiline comments
\usepackage{verbatim}


%making math environment work
\usepackage{amssymb,amsmath,amsfonts}
%chemistry packages
\usepackage{chemfig}
\usepackage[version=3]{mhchem}
%intelligent comma
\usepackage{icomma}
%siunits
\usepackage{siunitx}
%enhanced tables
\usepackage{tabularx}

%multiple columns
\usepackage{multicol}

%graphics
\usepackage{graphicx}
%controlling floats
\usepackage{placeins} %use with \FloatBarrier
\usepackage{float}
%\graphicspath{ {} }
\usepackage{pdfpages}
%code
\usepackage{fancyvrb}


%last!
%urls

\usepackage{hyperref}
%\hypersetup{colorlinks = true, urlcolor=blue, linkcolor=black}


%%%%%%%%%%%%%%%%%%%%%%%%%%%%%%%%%%%%%%%%%%%%%%%%%%%%%%%%%%%%%%%%%%%%
%%%%%%%%%%%%%%%%%%%%%%%%%%%%%%%%%%%%%%%%%%%%%%%%%%%%%%%%%%%%%%%%%%%%

\title{Introduction to {\LaTeX}}
\subtitle{A Method for Scientific Writing}
\date{\today}
\author{Margarita Tzivaki}
\institute{University of Ontario\\ Institute of Technology}
\logo{\includegraphics[width=.13\textwidth]{unilogo}}
\begin{document}
\fvset{formatcom=\color{CadetBlue}}


\frame{\maketitle}

\setlength{\parskip}{0.3cm}


%\AtBeginSection[]
%{
%  \begin{frame}<beamer>
%   \begin{multicols}{2}
%   %\begin{center}
%   \tableofcontents[hideothersubsections]
%   %\end{center}
%   \end{multicols}
%  \end{frame}
%}

%%%%%%%%%%%%%%%%%%%%%%%%%%%%%%%%%%%%%%%%%%%%%%%%%%%%%%%%%%%%%%%%%%%%
%%%%%%%%%%%%%%%%%%%%%%%%%%%%%%%%%%%%%%%%%%%%%%%%%%%%%%%%%%%%%%%%%%%%

\section*{Introduction}

\begin{frame}{Introduction}{Common Problems in Typesetting Documents with Word or LibreOffice}

\begin{figure}[H]
		\includegraphics[width=.5\textwidth]{versus}
\end{figure}

\end{frame}
%%%%%%%%%%%%%%%%%%%%%%%%%%%%%%%%%%%%%%%%%%%%%%%%%%%%%%%%%%%%%%%%%%%%
\begin{frame}
\frametitle{Introduction}\framesubtitle{Common Problems in Typesetting Documents with Word or LibreOffice}

\begin{itemize}
%\item Inconsistency of fonts and styles on different versions and computers.
\item Poor typographic control (kerning and leading, missing ligatures).% Result: Gaps, cramped letter spacing or inconsistent line heights
%\item Poor font control (Bold or italic instead of changing to the appropriate version of the font family).% Result: Italicized and bold text seems like it needs extra spacing.
%\item Missing proper numerals for chosen fonts.
\item "Badness" is difficult to resolve.
\item Image embedding (instead of external links) and limited image editing options.% Result: Image quality degrading after multiple saves 
\item Unexpected surprises when using external text.
\item Unprofessional look.
%\item Limited pdf support.% Result: Makes good looking prints really difficult
\item All those weird things that happen during auto-correction, style change, formatting and by just looking at it. %Result: Frustration 
\end{itemize}


\end{frame}
%%%%%%%%%%%%%%%%%%%%%%%%%%%%%%%%%%%%%%%%%%%%%%%%%%%%%%%%%%%%%%%%%%%%
\begin{frame}{Introduction}{What is {\LaTeX}?}

\textbf{{\TeX}} is a low-level markup and programming language created by Donald Knuth to typeset documents attractively and consistently.

\textbf{{\LaTeX}} is a macro package based on \TeX created by Leslie Lamport. Its purpose is to simplify TeX typesetting. Many later authors have contributed extensions, called packages or styles.
\end{frame}
%%%%%%%%%%%%%%%%%%%%%%%%%%%%%%%%%%%%%%%%%%%%%%%%%%%%%%%%%%%%%%%%%%%%
\begin{frame}
\frametitle{Introduction}\framesubtitle{Why {\LaTeX}?}
\begin{itemize}
\item {\LaTeX} is the standard for mathematical typesetting
\item {\LaTeX} is turning into the standard everywhere else and especially on the web (google docs, wordpress...)
\item {\LaTeX} is free (as in free speech not free beer)
\end{itemize}

\end{frame}
%%%%%%%%%%%%%%%%%%%%%%%%%%%%%%%%%%%%%%%%%%%%%%%%%%%%%%%%%%%%%%%%%%%%
\begin{frame}{Introduction}{Why {\LaTeX} for Me?}
\begin{itemize}
\item Separation of editing and processing
\item Sources are simple text files
\item Fast and easy uniformity
\item Emphasis on content
\item Facilitates collaborative working
\item Very good pdf support
\item Consistency and transparency of layouts and fonts
\item Easy typesetting for scientific requirements
\item Good handling of citations
\item You are forced to structure your documents correctly
\end{itemize}

\begin{center}
\textbf{AND...}

\end{center}
\end{frame}

%%%%%%%%%%%%%%%%%%%%%%%%%%%%%%%%%%%%%%%%%%%%%%%%%%%%%%%%%%%%%%%%%%%%
\begin{frame}{Introduction}{Why {\LaTeX} for Me?}

... It looks great.

\begin{figure}[H]
		\includegraphics[width=.35\textwidth]{latexr.jpg}
\end{figure}

\end{frame}

%%%%%%%%%%%%%%%%%%%%%%%%%%%%%%%%%%%%%%%%%%%%%%%%%%%%%%%%%%%%%%%%%%%%
\begin{frame}
\frametitle{Table of Contents}
\begin{center}
\begin{enumerate}
\setlength{\parskip}{0.5cm}
\item Some Quick Facts
\item Document Structure
\item Common Elements
\item Citing Literature
\item Advanced Topics
\end{enumerate}
\end{center}
\end{frame}
%%%%%%%%%%%%%%%%%%%%%%%%%%%%%%%%%%%%%%%%%%%%%%%%%%%%%%%%%%%%%%%%%%%%
\section{Some Quick Facts}
\subsection{Distributions and Editors}
\begin{frame}{Distributions and Editors}{Requirements}

\textbf{System:} The combination of the language and the macros. 
 
\textbf{Distribution:} The collection of packages and programs that enable you to typeset without having to manually fetch files and configure things.

\textbf{Engine:} An engine is an executable that can turn your source code to a printable output format. (pdflatex, latex) %The engine by itself only handles the syntax, it also needs to load fonts and macros to fully understand the source code and generate output properly. The engine will determine what kind of source code it can read, and what format it can ouput.


$\Rightarrow$ Distributions are an easy way to install what you need to use the engines and the systems you want.

\end{frame}
%%%%%%%%%%%%%%%%%%%%%%%%%%%%%%%%%%%%%%%%%%%%%%%%%%%%%%%%%%%%%%%%%%%%
\begin{frame}{Distributions and Editors}{Distributions}

%\begin{tabular}{l l}

\textbf{{\TeX}\textsc{Live}:} A  cross-platform {\TeX} distribution

\textbf{\textsc{Mac}{\TeX}:} A {\TeX}\textsc{Live} based distribution for Mac

\textbf{\textsc{MiK}{\TeX}:} A {\TeX} distribution for Windows

%\end{tabular}

\end{frame}
%%%%%%%%%%%%%%%%%%%%%%%%%%%%%%%%%%%%%%%%%%%%%%%%%%%%%%%%%%%%%%%%%%%%
\begin{frame}{Distributions and Editors}{Editors}
\begin{itemize}
\setlength{\parskip}{0.2cm}
\item \textbf{Cross-Platform:} \textcolor{red}{{\TeX}maker}, gedit (latex-plugin), {\TeX}works, Lyx (WYSIWYG), (Vim, emacs)
\item \textbf{Windows:} {\TeX}nicCenter, WinShell
\item \textbf{Linux:} Kile, {\LaTeX}ila, Gummi (WYSIWYG)
\item \textbf{Mac:} {\TeX}Shop,  {\TeX}nicle
\item \textbf{Web-based:}  {\LaTeX}Lab, Monkey{\TeX}
\end{itemize}
\end{frame}
%%%%%%%%%%%%%%%%%%%%%%%%%%%%%%%%%%%%%%%%%%%%%%%%%%%%%%%%%%%%%%%%%%%%
\subsection{Installation}
\begin{frame}{Installation}{On Windows, Linux and Mac}

\begin{itemize}
\setlength{\parskip}{0.2cm}
\item \textbf{Windows:} Install \textsc{MiK}{\TeX} or {\TeX}\textsc{Live}. \emph{After} that install your favourite editor. Configure the path to provide the editor with the exact location of the \textsc{MiK}{\TeX} software.
\item \textbf{Linux:} Make sure to have the full version of {\TeX}\textsc{Live}, install your favourite editor (use the package manager).
\item \textbf{Mac:} Install the \textsc{Mac}{\TeX} package (\textcolor{blue}{\url{http://tug.org/mactex/}}). Either use the editor that comes with it ({\TeX}Shop) or install your favourite editor.

\begin{block}
{Hint:}If \textsc{MiK}{\TeX} is not working try a different server.
\end{block}
\end{itemize}
\end{frame}


%%%%%%%%%%%%%%%%%%%%%%%%%%%%%%%%%%%%%%%%%%%%%%%%%%%%%%%%%%%%%%%%%%%%
%%%%%%%%%%%%%%%%%%%%%%%%%%%%%%%%%%%%%%%%%%%%%%%%%%%%%%%%%%%%%%%%%%%%

\section{Document Structure}

\subsection{Basics}
\begin{frame}[fragile]
\frametitle{Basics}\framesubtitle{Getting Started}

\begin{Verbatim}
\documentclass[a4paper]{article}

%my first hello world document
 
\begin{document}
 
hello world!
 
\end{document}
 
\end{Verbatim}

\end{frame}
%%%%%%%%%%%%%%%%%%%%%%%%%%%%%%%%%%%%%%%%%%%%%%%%%%%%%%%%%%%%%%%%%%%%
%\subsection{Basics}
\begin{frame}[fragile]
\frametitle{Basics}\framesubtitle{Environments, Commands, Comments}
 
\textbf{Commands:}\newline  \Verb+\command_name[option1,option2,...]{argument1}{argument2}+
 
\textbf{Comments:}\\ \Verb+% this is a comment+

\textbf{Environments:}\\ \Verb+\begin{environmentname} text influenced \end{environmentname}+
 
\textbf{Groups:}\\ \Verb+ { \command Inside the group.} Outside the group.+
 
 
\end{frame}
%%%%%%%%%%%%%%%%%%%%%%%%%%%%%%%%%%%%%%%%%%%%%%%%%%%%%%%%%%%%%%%%%%%%
\subsection{Components}
\begin{frame}[fragile]
\frametitle{Components}\framesubtitle{Document Format}


\textbf{Header}: Determines the formatting
\begin{itemize}
\item Document class: \textit{article}, \textit{book}, \textit{report}, \textit{letter} with options for fonts and printing (equivalent \textsc{KOMA} Skript classes: scr)\newline \Verb+\documentclass[options]{class}+
\item Usepackages: Activation of special macros \Verb+\usepackage[parameters]{package}+
\end{itemize}

\textbf{Main Body}: The content of the document that is being formatted by the header

\textbf{Special Pages}: Bibliography, appendix commands


\end{frame}

%%%%%%%%%%%%%%%%%%%%%%%%%%%%%%%%%%%%%%%%%%%%%%%%%%%%%%%%%%%%%%%%%%%%

\begin{frame}[fragile]
\frametitle{Components}\framesubtitle{Picking a Title}
\begin{Verbatim}
\documentclass[a4paper]{article}

\begin{document}

\title{My first LaTeX Document}
\author{You \and Me}
\date{\today}
\maketitle 

\end{document}
\end{Verbatim}
\end{frame}
%%%%%%%%%%%%%%%%%%%%%%%%%%%%%%%%%%%%%%%%%%%%%%%%%%%%%%%%%%%%%%%%%%%%
\begin{frame}[fragile]
\frametitle{Components}\framesubtitle{Writing an Abstract}
\begin{Verbatim}
\documentclass[a4paper]{article}

\begin{document}

\begin{abstract}
Your abstract goes here...
\end{abstract}

\end{document}
\end{Verbatim}
\end{frame}
%%%%%%%%%%%%%%%%%%%%%%%%%%%%%%%%%%%%%%%%%%%%%%%%%%%%%%%%%%%%%%%%%%%%
\begin{frame}[plain]
%\frametitle{Components}\framesubtitle{Title and Abstract}
\begin{figure}[H]
\vspace*{-1cm}\hspace*{-1.5cm}
		\includegraphics[width=1.2\textwidth]{title.pdf}
\end{figure}
\end{frame}
%%%%%%%%%%%%%%%%%%%%%%%%%%%%%%%%%%%%%%%%%%%%%%%%%%%%%%%%%%%%%%%%%%%%
\begin{frame}[fragile]
\frametitle{Components}\framesubtitle{Main Body and Table of Contents}
\begin{Verbatim}
\begin{document}

\tableofcontents

\section{Title of the First Section}
... text ... 
\subsection{Title of the First Subsection}
... text ... 
\subsubsection{Title of the First Subsubsection}
... text ... 
\subsubsection*{Title of the Second Subsubsection} 
\addcontentsline{toc}{subsubsection}{Something Else} 

\end{document}
\end{Verbatim}
\end{frame}
%%%%%%%%%%%%%%%%%%%%%%%%%%%%%%%%%%%%%%%%%%%%%%%%%%%%%%%%%%%%%%%%%%%%
\begin{frame}[plain]
%\frametitle{Components}\framesubtitle{Main Body and Table of Contents}
\begin{figure}[H]
\vspace*{-1cm}\hspace*{-1.5cm}
		\includegraphics[width=1.1\textwidth]{tableofcontents.pdf}
\end{figure}
\end{frame}
%%%%%%%%%%%%%%%%%%%%%%%%%%%%%%%%%%%%%%%%%%%%%%%%%%%%%%%%%%%%%%%%%%%%
\subsection{Compiling {\LaTeX}}
\begin{frame}{Compiling {\LaTeX}}{}

\begin{figure}[H]
		\includegraphics[width=.4\textwidth]{compilinglatex}
\end{figure}

\begin{alertblock}
{Warning!}The only important file types are .tex, .cls and .sty, .bib and .bst. They are not temporary and should not be deleted.
\end{alertblock}

\end{frame}
%%%%%%%%%%%%%%%%%%%%%%%%%%%%%%%%%%%%%%%%%%%%%%%%%%%%%%%%%%%%%%%%%%%%
\subsection{Layout}
\begin{frame}[fragile]
\frametitle{Layout}\framesubtitle{Page Style}

\Verb+\pagestyle{'style'}+ and \Verb+\thispagestyle{'style'}+

\begin{itemize}
\item \textbf{empty:} Header and footer are cleared
\item \textbf{plain:} Header is clear, but the footer contains the page number in the center
\item \textbf{headings:} Footer is blank, header displays information according to document class and page number top right
\item \textbf{myheadings:} Page number is top right, and it is possible to control the rest of the header
\item \textbf{fancy:} For better control over the headers and footers \Verb+\usepackage{fancyhdr}+
\end{itemize}


\end{frame}
%%%%%%%%%%%%%%%%%%%%%%%%%%%%%%%%%%%%%%%%%%%%%%%%%%%%%%%%%%%%%%%%%%%%
\subsection{Formatting}
\begin{frame}[fragile]
\frametitle{Formatting}\framesubtitle{Page Size and Structure}
\begin{itemize}
\item {\LaTeX} comes with predefined page and margin sizes for every style and document class
\item For manipulation: \Verb+\usepackage[options]{geometry}+
\item The landscape format is an option of the geometry package
\item Text in multiple columns: \Verb+\begin{multicols}{#}...lots of text...\end{multicols}+
\end{itemize}
\begin{alertblock}
{Warning!}As {\LaTeX} is a globally recognized set of typesetting defaults, additional page formatting should be done not without reason and always with great care.
\end{alertblock}

\end{frame}
%%%%%%%%%%%%%%%%%%%%%%%%%%%%%%%%%%%%%%%%%%%%%%%%%%%%%%%%%%%%%%%%%%%%

\begin{frame}[fragile]\frametitle{Formatting}\framesubtitle{Colors}

\Verb+\usepackage{color}+ and \Verb+\usepackage[options]{xcolor}+

\begin{itemize}
\item Define colors: \Verb+\definecolor{'name'}{'model'}{'color-spec'}+
\item Coloring text: \Verb+\textcolor{declared-color}{text}+
\item Coloring the background: \Verb+\colorbox{declared-color}{text}+
\end{itemize}

\end{frame}

%%%%%%%%%%%%%%%%%%%%%%%%%%%%%%%%%%%%%%%%%%%%%%%%%%%%%%%%%%%%%%%%%%%%

\begin{frame}[fragile]\frametitle{Formatting}\framesubtitle{Fonts}

Various font styles, shapes and sizes are available.

Font encoding can be modified with \Verb+\usepackage['encoding']{fontenc}+

\begin{alertblock}
{Warning!}For the sake of consistent typography playing a lot with fonts is highly discouraged. This is the work of font and class designers, not end users.
\end{alertblock}

\end{frame}

%%%%%%%%%%%%%%%%%%%%%%%%%%%%%%%%%%%%%%%%%%%%%%%%%%%%%%%%%%%%%%%%%%%%
%%%%%%%%%%%%%%%%%%%%%%%%%%%%%%%%%%%%%%%%%%%%%%%%%%%%%%%%%%%%%%%%%%%%

\section{Common Elements}
\begin{frame}{Common Elements}{}

\begin{columns}
\begin{column}[t]{5cm}
\vspace*{1.cm}
\chemfig{C(-[:0]H)(-[:90]H)(-[:180]H)(-[:270]H)}\newline 

\ce{CO2 + C -> 2CO}\newline

\begin{equation}
\exp(x) = \sum_{n = 0}^\infty \frac{x^n}{n!} = \lim_{n \to \infty} \left( 1 + \frac{x}{n} \right)^n \nonumber
\end{equation}

\end{column}
\begin{column}[t]{5cm}

\begin{tabular}{|r|l|}
  \hline
  7C0 & hexadecimal \\
  3700 & octal \\ \cline{2-2}
  11111000000 & binary \\
  \hline \hline
  1984 & decimal \\
  \hline
\end{tabular}
\vspace*{1.cm}
\begin{enumerate}
\item first item
\item second item
\end{enumerate}


\end{column}
\end{columns}

\end{frame}
%%%%%%%%%%%%%%%%%%%%%%%%%%%%%%%%%%%%%%%%%%%%%%%%%%%%%%%%%%%%%%%%%%%%
\subsection{Lists}
\begin{frame}[fragile]\frametitle{Lists}\framesubtitle{Sorted and Unsorted}

sorted lists: \Verb+\begin{enumerate}+ \Verb+\item[]+ \Verb+\end{enumerate}+
\begin{enumerate}
\item first item
\item second item
\end{enumerate}
unsorted lists: \Verb+\begin{itemize}+ \Verb+\item[]+ \Verb+\end{itemize}+
\begin{itemize}
\item first item
\item second item
\end{itemize}

\begin{comment}
\begin{columns}
\begin{column}[t]{5cm}

\end{column}
\begin{column}[t]{5cm}

\end{column}
\end{columns}
\end{comment}

\end{frame}
%%%%%%%%%%%%%%%%%%%%%%%%%%%%%%%%%%%%%%%%%%%%%%%%%%%%%%%%%%%%%%%%%%%%
\subsection{Floats}

\begin{frame}[fragile]\frametitle{Floats}\framesubtitle{Concept and Problem-Solving}

Floats are containers for things in a document that cannot be broken over a page $\rightarrow$ they float (graphs, tables)
\begin{itemize}

\item Placement specifiers \Verb+\begin{float}[h!,t,b]+
\item \Verb+\usepackage{float}+ provides the placement specifier \Verb+[H]+
\item \Verb+\usepackage{placeins}+ use with \Verb+\FloatBarrier+
\end{itemize}

\begin{block}
{Hint:} If many floats occur in rapid succession, {\LaTeX} stacks them all up and prints them together or leaves them to the end in protest. 
\end{block}
\end{frame}

%%%%%%%%%%%%%%%%%%%%%%%%%%%%%%%%%%%%%%%%%%%%%%%%%%%%%%%%%%%%%%%%%%%

\begin{frame}[fragile]\frametitle{Floats}\framesubtitle{Formatting Tables}
\begin{itemize}
\item Tabular environment:\newline\Verb+\begin{tabular}[pos]{table spec}...\end{tabular}+
\item Tabular commands: For more control over tables: \Verb+\usepackage{tabularx}+, \Verb+\usepackage{booktabs}+, \Verb+\usepackage{tabu}+, 
\item Introducing tables in float environment: \Verb+\begin{table}...tabular...\end{table}+

\end{itemize}

\begin{columns}
\begin{column}[t]{1.cm}
\end{column}
\begin{column}[t]{6cm}
%\textbackslash begin\{center\}\newline
\begin{Verbatim}
 \begin{tabular}{ l | c | r }
    1 & 2 & 3 \\ \hline
    4 & 5 & 6 \\ \hline
 \end\{tabular\}}
  \end{Verbatim}
%\textbackslash end\{center\}}
\end{column}
\begin{column}[t]{7cm}
\begin{center}
\begin{tabular}{ l | c | r }
    1 & 2 & 3 \\
    \hline
    4 & 5 & 6 \\
  \end{tabular}
  \end{center}
\end{column}
\end{columns}
\end{frame}
%%%%%%%%%%%%%%%%%%%%%%%%%%%%%%%%%%%%%%%%%%%%%%%%%%%%%%%%%%%%%%%%%%%

\begin{frame}[fragile]\frametitle{Floats}\framesubtitle{Graphics: Import and Placement}

\begin{itemize}
%\item The filetypes used depend on the method of compilation.

\item \Verb+\usepackage{graphicx}+ \newline \Verb+\graphicspath{{'path'}}+

\item Insert files in text:  \Verb+\includegraphics*[parameters]{mypicture}+

\item Introducing graphics in float environment: \Verb+\begin{figure}...graphics...\end{figure}+
\end{itemize}
\begin{block}
{Hint:}You should always prefer vector graphics if possible (EPS, PDF).
\end{block}

\end{frame}
%%%%%%%%%%%%%%%%%%%%%%%%%%%%%%%%%%%%%%%%%%%%%%%%%%%%%%%%%%%%%%%%%%%
\begin{frame}[fragile]\frametitle{Floats}\framesubtitle{Including Pictures}

\begin{Verbatim}

\begin{figure}[htb]
\centering
\includegraphics[width=0.8\textwidth]{image.png}
\caption{Awesome Image}
\label{fig:awesome_image}
\end{figure}

\end{Verbatim}

\end{frame}

%%%%%%%%%%%%%%%%%%%%%%%%%%%%%%%%%%%%%%%%%%%%%%%%%%%%%%%%%%%%%%%%%%%
\subsection{Mathematics}
\begin{frame}[fragile]
\frametitle{Mathematical Symbols}\framesubtitle{Symbols and Equations}

\Verb+\usepackage{amsmath}+

\begin{itemize}
\item Math environment: \Verb+\begin{equation}...equation...\end{equation}+

\item Inline math environment: \Verb+$...equation...$+
\end{itemize}

\begin{columns}
\begin{column}[t]{6cm}
\centering
\vspace*{0.25cm}
\newline
\Verb+\frac{\alpha^2}{\beta^2}+
\end{column}
\begin{column}[t]{3cm}
\begin{equation}
\frac{\alpha^2}{\beta^2} \nonumber
\end{equation}
\end{column}
\end{columns}

%\begin{block}
%{Hint} 
\textbf{Manage correct spacing for units} \Verb+\usepackage{siunitx}+ is used with \Verb+\SI{'number'}{'unit'}+\newline
%\end{block}

\end{frame}
%%%%%%%%%%%%%%%%%%%%%%%%%%%%%%%%%%%%%%%%%%%%%%%%%%%%%%%%%%%%%%%%%%%
\subsection{Chemistry}
\begin{frame}[fragile]\frametitle{Chemical Equations}\frametitle{Constitutional Formulas and Equations}

\Verb+\usepackage{chemfig}+ and \Verb+\usepackage[version=3]{mchem}+
\begin{itemize}
\item Chemical Graphics: \Verb+\chemfig{<atom1><bond type>[parameters]<atom2>}}+\newline

\begin{columns}
\begin{column}[t]{5cm}
\centering
\vspace*{0.25cm}
\Verb+\chemfig{A=B}+
\end{column}
\begin{column}[t]{5cm}
\chemfig{A=B}
\end{column}
\end{columns}


\item Chemical Equations: \Verb+\ce{...equation...}+\newline 
\begin{columns}
\begin{column}[t]{5cm}
\centering
\vspace*{0.25cm}
\Verb+\ce{C02+\textcolor{CadetBlue}{$+$}\Verb+C+\textcolor{CadetBlue}{$->$}\Verb+2CO2}+
%\Verb+\ce{CO2\+C\-\>2CO}+
\end{column}
\begin{column}[t]{5cm}
\ce{CO2 + C -> 2CO}
\end{column}
\end{columns}
\end{itemize}
\end{frame}

%%%%%%%%%%%%%%%%%%%%%%%%%%%%%%%%%%%%%%%%%%%%%%%%%%%%%%%%%%%%%%%%%%%%
%%%%%%%%%%%%%%%%%%%%%%%%%%%%%%%%%%%%%%%%%%%%%%%%%%%%%%%%%%%%%%%%%%%%

\section{Citing Literature}

\begin{frame}{Citing Literature} {\textsc{Bib}{\TeX}}%{How does it work?}
BibTeX provides for the storage of all references in an external, flat-file database.
\begin{itemize}
\item \textbf{Environment:} \textcolor{CadetBlue}{\tt \textbackslash bibliography\{bibfile\}}
\item \textbf{Two options:} \begin{itemize}
						\item Type every entry manually
						\item Use a database that produces \textsc{Bib}{\TeX} code (strongly recommended!)
						\end{itemize}
\end{itemize}

\end{frame}

\subsection{Literature Databases}
\begin{frame}{Literature Databases}{}
\begin{itemize}
\item JabRef
\item EndNote (does not import \textsc{Bib}{\TeX})
\item Citavi
\item Mendeley
\item CiteULike
\item RefWorks (web based)
\end{itemize}

Check out also \textcolor{blue}{\url{http://en.wikipedia.org/wiki/Comparison_of_reference_management_software}} for the complete list.

\begin{block}
{Hint:} Google scholar and most paper-search websites (like SciVerse) can export \textsc{Bib}{\TeX} entries.
\end{block}

\end{frame}

\subsection{Citing}
\begin{frame}[fragile]\frametitle{Citing}\framesubtitle{Basics and Styles}

\Verb+\bibliographystyle{style}+ \newline \Verb+\bibliography{mybibliography1,mybibliography2}+

\begin{itemize}
	\item[] Various styles available: plain, abstract, named ...
  \item[] Standard {\LaTeX} bibliography: numeric style of citations
  \item[] For alternative options (journal or research specific) use the package: \Verb+\usepackage[options]{natbib}+
	
\end{itemize}
%
%\begin{alertblock}
%{Warning!} If you use the natbib package, don't forget to adjust your style selection to the corresponding natbib style.
%\end{alertblock}
\end{frame}


\begin{frame}[fragile]\frametitle{Citing}\framesubtitle{\textsc{Bib}{\TeX} Entries}

 \Verb+\cite{citation_keyl}+\newline
 \Verb+\cite{citation01,citation02,citation03}+

\textbf{ \textsc{Bib}{\TeX} entry:}
\begin{verbatim}
@article{greenwade93,
    author  = "George D. Greenwade",
    title   = "The {C}omprehensive {T}ex {A}rchive
    {N}etwork ({CTAN})",
    year    = "1993",
    journal = "TUGBoat",
    volume  = "14",
    number  = "3",
    pages   = "342--351"
 }
\end{verbatim}
\end{frame}

%%%%%%%%%%%%%%%%%%%%%%%%%%%%%%%%%%%%%%%%%%%%%%%%%%%%%%%%%%%%%%%%%%%%

\section{Advanced Topics}

\begin{frame}{Advanced Topics}{What now?}

Now the interesting part begins! 5 reasons to use {\LaTeX} in a scientific environment.

\begin{columns}
\begin{column}[t]{5cm}
\begin{itemize}
\item Special documents
\begin{itemize}
\item Presentation
\item Poster
\item CV and cover letter
\item Teaching stuff
\end{itemize}
\end{itemize}

\end{column}
\begin{column}[t]{5cm}
\begin{itemize}
\item Modular documents
\item Version control
\item Controlling external graphs
\item Creating graphics

\end{itemize}
\end{column}
\end{columns}



\end{frame}

%%%%%%%%%%%%%%%%%%%%%%%%%%%%%%%%%%%%%%%%%%%%%%%%%%%%%%%%%%%%%%%%%%%%
\subsection{Special Documents}
\subsubsection{Presentations}
\begin{frame}[fragile]\frametitle{Special Documents}\framesubtitle{Presentation}

\Verb+\documentclass{beamer}+

{\LaTeX} provides various themes along with colors:\newline
\Verb+\usetheme{'theme'}+ and \Verb+\usecolortheme{'theme'}+

Additional to the traditional sections hierarchy, beamer class comes with "frames" corresponding to the individual slides.
\Verb+\begin{frame}...text...\end{frame}+\newline

\begin{block}
{Hint:} At \textcolor{blue}{\url{http://deic.uab.es/~iblanes/beamer_gallery/index.html}} all available basic themes can be looked up.
\end{block}


\end{frame}
%%%%%%%%%%%%%%%%%%%%%%%%%%%%%%%%%%%%%%%%%%%%%%%%%%%%%%%%%%%%%%%%%%%%
\subsubsection{Posters}
\begin{frame}[fragile]\frametitle{Special Documents}\framesubtitle{Poster}

\Verb+\usepackage[orientation,size,scale]{beamerposter}+

\begin{itemize}
\item Flexibility of fonts and sizes
\item Flexibility of orientation
\item Beamer themes
\item Textblocks: \Verb+textpos+ package for positioning control
\end{itemize}




\begin{block}
{Hint:}\textcolor{blue}{\url{http://tug.org/pracjourn/2012-1/shang/shang.pdf}} is a great introduction.
\end{block}


\end{frame}
%%%%%%%%%%%%%%%%%%%%%%%%%%%%%%%%%%%%%%%%%%%%%%%%%%%%%%%%%%%%%%%%%%%%
\begin{frame}[plain]%{Special Documents}{Poster}

\begin{figure}[H]
		\includegraphics[width=1.\textwidth]{posterfinal.pdf}
\end{figure}

\end{frame}

%%%%%%%%%%%%%%%%%%%%%%%%%%%%%%%%%%%%%%%%%%%%%%%%%%%%%%%%%%%%%%%%%%%%

%%%%%%%%%%%%%%%%%%%%%%%%%%%%%%%%%%%%%%%%%%%%%%%%%%%%%%%%%%%%%%%%%%%%
\subsubsection{CV}
\begin{frame}[fragile]\frametitle{Special Documents}\framesubtitle{Curriculum Vitae}

\Verb+\documentclass[options]{moderncv}+\newline
\Verb+\moderncvstyle{"style"}+ \newline
\Verb+\moderncvcolor{"color"}+


\begin{columns}
\begin{column}[t]{5cm}
\begin{figure}[H]
		\includegraphics[width=.95\textwidth]{moderncv}
\end{figure}
\end{column}
\begin{column}[t]{5cm}
\begin{figure}[H]
		\includegraphics[width=.95\textwidth]{moderncv2}
\end{figure}
\end{column}
\end{columns}


\end{frame}
%%%%%%%%%%%%%%%%%%%%%%%%%%%%%%%%%%%%%%%%%%%%%%%%%%%%%%%%%%%%%%%%%%%%
\subsection{Modular Documents}
\begin{frame}[fragile]\frametitle{Modular Documents}\framesubtitle{Introducing Order}

\begin{itemize}
\item[] Very useful strategy for long {\LaTeX} documents: split in several files.
\item[] Best practice: \begin{itemize}
						\item main document (main.tex)
						\item style document (style.sty)
						\item latex files folder
						\item pictures folder \end{itemize}
\item[] include documents with \Verb+\include{filename}+
\end{itemize}

\begin{block}
{Hint:} To compile the child documents separate from the mother document use \Verb+\usepackage{subfiles}+.
\end{block}



\end{frame}
%%%%%%%%%%%%%%%%%%%%%%%%%%%%%%%%%%%%%%%%%%%%%%%%%%%%%%%%%%%%%%%%%%%%
\subsection{More Options}
\begin{frame}
\frametitle{More Options}\framesubtitle{... for more convenience!}
\begin{itemize}
\item More Special Documents: letters, cover letters, exams, assignments
\item Version control: backups, collaborative work, non-destructive editing
\item External graphs typesetting: control gnuplot graphs
\item Creating graphics: with the \Verb+tikz+ package
\end{itemize}

\end{frame}
%%%%%%%%%%%%%%%%%%%%%%%%%%%%%%%%%%%%%%%%%%%%%%%%%%%%%%%%%%%%%%%%%%%%
\subsection{Troubleshooting}
\begin{frame}
\frametitle{Troubleshooting}\framesubtitle{What to do if it just doesn't work}

\begin{enumerate}
\item Check the log file for a detailed error message or line number
\item Check for missing or surplus brackets
\item Check for problems in closing an environment
\item Delete all temporary files and compile again
\item Copy and paste the error message in your browser
\item For Mik{\TeX} related issues: Don't start installing packages manually unless you are sure you know what you are doing!
\item If Mik{\TeX} still doesn't work: Use {\TeX}Live
\end{enumerate}

\end{frame}

%%%%%%%%%%%%%%%%%%%%%%%%%%%%%%%%%%%%%%%%%%%%%%%%%%%%%%%%%%%%%%%%%%%%
%%%%%%%%%%%%%%%%%%%%%%%%%%%%%%%%%%%%%%%%%%%%%%%%%%%%%%%%%%%%%%%%%%%%
\section*{Questions}

\begin{frame}{Questions?}{}

\begin{figure}[H]
		\includegraphics[width=.25\textwidth]{question-mark}
\end{figure}

\end{frame}

\begin{frame}{... and Answers}{}
\begin{itemize}
\item The not so short introduction: \textcolor{blue}{\url{http://tobi.oetiker.ch/lshort/lshort.pdf}}
\item A great book: \textcolor{blue}{\url{https://en.wikibooks.org/wiki/Latex}}
\item Forum for any kind of problem and any kind of solution: \textcolor{blue}{\url{http://tex.stackexchange.com/}}
\item The {\TeX} Archive Network \textcolor{blue}{\url{http://www.ctan.org/}}
\item The {\LaTeX} Community: \textcolor{blue}{\url{http://www.latex-community.org/}}
\item De{\TeX}ify: \textcolor{blue}{\url{http://detexify.kirelabs.org/classify.html}}
\end{itemize}


\end{frame}

\begin{comment}
\begin{frame}{What now?}{If you are still not satisfied with the options...}

... you can always write your own packages!

\end{frame}
\end{comment}

%%%%%%%%%%%%%%%%%%%%%%%%%%%%%%%%%%%%%%%%%%%%%%%%%%%%%%%%%%%%%%%%%%%%
%%%%%%%%%%%%%%%%%%%%%%%%%%%%%%%%%%%%%%%%%%%%%%%%%%%%%%%%%%%%%%%%%%%%



\end{document}
